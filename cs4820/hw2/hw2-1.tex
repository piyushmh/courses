\documentclass[12pt]{article}
%\usepackage{fullpage}
\usepackage{epic}
\usepackage{eepic}
\usepackage{paralist}
\usepackage{graphicx}
\usepackage{algorithm,algorithmicx}
\usepackage[noend]{algpseudocode}
\newcommand*\Let[2]{\State #1 $\gets$ #2}

\usepackage{tikz}
\usepackage{xcolor,colortbl}
\usepackage{amsmath,amsthm,amssymb}

%\newcommand*\Let[2]{\State #o1 $\gets$ #2}
\newtheorem{theorem}{Theorem}
\newtheorem{lemma}[theorem]{Lemma}
%\newtheorem{proposition}[theorem]{Proposition}
%\newtheorem{corollary}[theorem]{Corollary}

\newenvironment{definition}[1][Definition]{\begin{trivlist}
\item[\hskip \labelsep {\bfseries #1}]}{\end{trivlist}}
\newenvironment{example}[1][Example]{\begin{trivlist}
\item[\hskip \labelsep {\bfseries #1}]}{\end{trivlist}}
\newenvironment{remark}[1][Remark]{\begin{trivlist}
\item[\hskip \labelsep {\bfseries #1}]}{\end{trivlist}}

\newcommand*\Th[1]{{#1}^{\textrm{th}}}

%%%%%%%%%%%%%%%%%%%%%%%%%%%%%%%%%%%%%%%%%%%%%%%%%%%%%%%%%%%%%%%%
% This is FULLPAGE.STY by H.Partl, Version 2 as of 15 Dec 1988.
% Document Style Option to fill the paper just like Plain TeX.

\typeout{Style Option FULLPAGE Version 2 as of 15 Dec 1988}

\topmargin 0pt
\advance \topmargin by -\headheight
\advance \topmargin by -\headsep

\textheight 8.9in

\oddsidemargin 0pt
\evensidemargin \oddsidemargin
\marginparwidth 0.5in

\textwidth 6.5in
%%%%%%%%%%%%%%%%%%%%%%%%%%%%%%%%%%%%%%%%%%%%%%%%%%%%%%%%%%%%%%%%

\pagestyle{empty}
\setlength{\oddsidemargin}{0in}
\setlength{\topmargin}{-0.8in}
\setlength{\textwidth}{6.8in}
\setlength{\textheight}{9.5in}

\setcounter{secnumdepth}{0}

\setlength{\parindent}{0in}
\addtolength{\parskip}{0.2cm}
\setlength{\fboxrule}{.5mm}\setlength{\fboxsep}{1.2mm}
\newlength{\boxlength}\setlength{\boxlength}{\textwidth}
\addtolength{\boxlength}{-4mm}

\newcommand{\algobox}[2]{
  \begin{center}
    \framebox{\parbox{\boxlength}{
        \textbf{Introduction to Algorithms} \hfill \textbf{#1}\\
        \textbf{CS 4820, Spring 2014} \hfill \textbf{#2}}}
  \end{center}}

\newcommand{\algosolutionbox}[2]{
  \begin{center}
    \framebox{\parbox{\boxlength}{
        \textbf{CS 4820, Spring 2014} \hfill \textbf{#1}\\
        #2
      }}
  \end{center}}


\begin{document}

\algosolutionbox{Homework 2, Problem 1}{
  % TODO: fill in your own name, netID, and collaborators
  Name: Piyush Maheshwari\\
  NetID: pm489\\
  Collaborators: None
}

\bigskip

\textbf{(1)}
\emph{(10 points)}
Consider the following candidate algorithm for computing
a minimum weight spanning tree of a graph with $n$ vertices
and $m$ edges.

The algorithm builds a subgraph proceeding in rounds and
starting from the empty subgraph.
In each round,  for each
connected component of the current subgraph,
the algorithm selects an edge of
minimum weight that leaves the component.
At the end of the round it adds these selected edges to
the subgraph. The algorithm stops when the subgraph is connected.

\textbf{(1.a)}
Show that the subgraph computed by the algorithm is a minimum weight spanning tree of the input graph.
You may assume that the input graph is connected and that all edge weights are distinct.

\textbf{(1.b)}
Show how to implement the algorithm in time $O(m \log n)$.

{\em \textbf{Warning:}
This minimum spanning tree algorithm is moderately well known,
and it is possible to find analyses of it on the Web.
If you use Web sources you must cite them, and citing a Web
source that contains an analysis of this algorithm will
result in loss of points. For this reason, you are advised
to solve the problem using your textbook and lecture notes
(and office hours or Piazza, if desired), but not consulting
the Web.}

\subsection{Solutions}
\textbf{(1.a)}

Let the input graph be denoted by $G=(V,E)$. Let the sub-graph be $T$. Initially $T =  \{V , \emptyset \}$.

%\setcounter{lemma}{1}
\begin{lemma} 
The minimum weight edge leaving a component as selected by the algorithm is a minimum weight edge crossing some cut S of graph G.
\end{lemma}
\begin{proof}
Suppose at any point in the algorithm the sub-graph has components $c_1, c_2, ... c_i$. Now we choose a minimum weight edge leaving component $c_j$. Let this edge be $(v,w)$. The vertex $v$ lies in component $c_j$. The vertex $w$ will lie in some component $c_k \neq c_j$ since $(v,w)$ leaves the component $c_j$. We know that $V-c_j$ is not empty otherwise the algorithm would have terminated. Consider the cut $c_j$ and $V- c_j$. Clearly edge $(v,w)$ is lowest cost edge crossing the cut since $(v,w)$ chosen to be the minimum cost edge with one vertex in $c_j$ and one in $V-c_j$. We can prove this by contradiction as well. Suppose some other edge $(v',w')$ is the minimum cost edge crossing the cut $c_j$. This means that edge $(v',w')$ is the minimum cost edge leaving the cut. But we assumed earlier that $(v,w)$ is the minimum cost edge leaving the component $c_j$. Hence this is contradiction which proves that the edge selected by the algorithm $(v,w)$ is also the minimum edge cost crossing the cut $c_j$.
\end{proof}
Now the algorithm in each round picks up minimum cost edge leaving each connected component. Using Lemma 1 we know that the every such edge chosen is a minimum edge crossing some cut of the graph. Now using the cut property of MST, we know that these edges chosen by the algorithm would the part of every MST possible. Also since we only pick up edges which connect disconnected components and we stop once the sub graph is connected, we can never have cycles in the sub graph. This means that when the algorithm terminates we will have a spanning tree. Hence the sub graph constructed by this algorithm is a minimum spanning tree of the input graph.

\textbf{(1.b)}

\subsection{Algorithm}

The algorithm starts out by running MakeUnion(V). Initially we make a priority queue for each connected component. Each priority queue contains edges coming out of that In each round we go through each of the connected components that we have, take out an edge from its priority queue and add that to MST that we are building. Then we merge the priority queues of the connected components that the graphs connect. For implementation purposes we can maintain a list of component components that keep decreasing in count and maintain the corresponding priority queue for each component.

\begin{algorithm}[H]
  \caption{FIND-MST}
  \begin{algorithmic}[1]
	\Function{FIND-MST}{G}
		\State $G' =  \{V , \emptyset \}$.
		\State Make-Union(n) \Comment{Make a component for each vertex}
		\State Make-PriorityQueue(N) \Comment{Make a PQ for each vertex by adding edges for that vertex}
		\While{$G'$ is still not connected}
			\Let {$t \left[\right]$}{$\emptyset$}
			\ForAll{$c_i$ in components}
				\State Extract min element ${u,v}$ from PQ of $c_i$
				\State Add $u,v$ to $t\left[\right]$
			\EndFor
			\ForAll{$u,v$ in $t\left[\right]$}
				\State Add $u,v$ to $G'$
				\State $x \gets$ find(u)
				\State $y \gets$ find(v)
				\State Union(x,y)
				\State Merge-PQ($x,y$)
			\EndFor
		\EndWhile
		\State \Return{$S$}
	\EndFunction
  \end{algorithmic}
\end{algorithm}


If we maintain priority-queues by simple lists, the following operations can be performed as follows - 

Extract Min - $O(1)$, Pick the head of the list\\
Merge-PQ of size m, n -  $O(m+n)$ , since this is same as merging two sorted lists of size m and n \\
Make-PQ of size n - $O(nlogn)$, this is simply sorting and storing the list.\\

If we represent union find data structures using forest representation as discussed in class\\
find (v) = $O(logn)$\\
union(x,y) = $O(logn)$\\

\subsection{Complexity Analysis}

\begin{enumerate}
\item Step 3 takes $\textbf{O(n)}$
\item Step 4 takes $O(n_1logn_1) + O(n_2logn_2) ..O(n_nlogn_n)$ where $n_1 + n_2 .. n_n = $m (edge list size)
This can be asymptotically bounded by $\textbf{O(mlogm)}$
\item Step 12,13,14 happens one for every edge since one an edge has been taken out of the PQ it is not added again. Hence these take $\textbf{3mlog(n)}$
\item Step 14 - There is a subtle issue with merging of PQ's. Since a priority queue for a component contains edges that leave that component, merging two PQ of different components might have edges which go to each other. So the finally merge PQ will have edges that have both end points within the same component. We can handle this issue while doing extract min element from PQ. Whenever we do an extract, if that edge $u,v$ has find(u)= find(v), then we discard that edge and pick the next edge. So for now, we can simple assume that merging lists take linear time. Now we can observe that the size of a component increases by atleast half for the component of lower size. This means that the maximum number of rounds that we can have are $log(n)$. In each round we can have many components merging. But since each merge is linear, if we sum them together they take $O(m)$ time. Hence the total time spent in merge would be $\textbf{mlogn}$.
\item Step 8- In normal case extract min is $O(1)$ but in this case when have to do a find operation for both end points as explained in 4 above. We do a extract-min once for every edge since once an edge has been taken out of the PQ, it is never added again. Hence the total time this operation takes is $\textbf{0(2mlogn)}$
\end{enumerate}

Total running time  = $\textbf{O(n)}$ + $\textbf{O(mlogm)}$ + $\textbf{3mlog(n)}$ + $\textbf{mlogn}$ + $\textbf{0(2mlogn)}$ = $\textbf{O(mlogn)}$ using the fact that log(n) is equal to log(m) asymptotically.

\end{document}
