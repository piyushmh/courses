\documentclass[12pt]{article}
 
\usepackage{fullpage}
\usepackage{amsmath,amsthm,amssymb}
\usepackage{algorithmicx, algorithm}
\usepackage[noend]{algpseudocode}

\newcommand*\Let[2]{\State #1 $\gets$ #2}
\newtheorem{theorem}{Theorem}
\newtheorem{lemma}[theorem]{Lemma}
\newtheorem{proposition}[theorem]{Proposition}
\newtheorem{corollary}[theorem]{Corollary}

\newenvironment{definition}[1][Definition]{\begin{trivlist}
\item[\hskip \labelsep {\bfseries #1}]}{\end{trivlist}}
\newenvironment{example}[1][Example]{\begin{trivlist}
\item[\hskip \labelsep {\bfseries #1}]}{\end{trivlist}}
\newenvironment{remark}[1][Remark]{\begin{trivlist}
\item[\hskip \labelsep {\bfseries #1}]}{\end{trivlist}}

\newcommand*\Th[1]{{#1}^{\textrm{th}}}

\setlength{\parindent}{0pt}
%\renewcommand{\algorithmicforall}{\textbf{for each}}
%%%%%%%%%%%%%%%%%%%%%%%%%%%%%%%%%%%%%%%%%%%%%%%%%%%%%%%%%%%%%%%%
% This is FULLPAGE.STY by H.Partl, Version 2 as of 15 Dec 1988.
% Document Style Option to fill the paper just like Plain TeX.

\typeout{Style Option FULLPAGE Version 2 as of 15 Dec 1988}

\topmargin 0pt
\advance \topmargin by -\headheight
\advance \topmargin by -\headsep

\textheight 8.9in

\oddsidemargin 0pt
\evensidemargin \oddsidemargin
\marginparwidth 0.5in

\textwidth 6.5in
%%%%%%%%%%%%%%%%%%%%%%%%%%%%%%%%%%%%%%%%%%%%%%%%%%%%%%%%%%%%%%%%

\pagestyle{empty}
\setlength{\oddsidemargin}{0in}
\setlength{\topmargin}{-0.8in}
\setlength{\textwidth}{6.8in}
\setlength{\textheight}{9.5in}

\setcounter{secnumdepth}{0}

\setlength{\parindent}{0in}
\addtolength{\parskip}{0.2cm}
\setlength{\fboxrule}{.5mm}\setlength{\fboxsep}{1.2mm}
\newlength{\boxlength}\setlength{\boxlength}{\textwidth}
\addtolength{\boxlength}{-4mm}

\newcommand{\algobox}[2]{
  \begin{center}
    \framebox{\parbox{\boxlength}{
        \textbf{Introduction to Algorithms} \hfill \textbf{#1}\\
        \textbf{CS 4820, Spring 2014} \hfill \textbf{#2}}}
  \end{center}}

\newcommand{\algosolutionbox}[2]{
  \begin{center}
    \framebox{\parbox{\boxlength}{
        \textbf{CS 4820, Spring 2014} \hfill \textbf{#1}\\
        #2
      }}
  \end{center}}


\begin{document}

\algosolutionbox{Homework 2, Problem 1}{
  % TODO: fill in your own name, netID, and collaborators
  Name: Piyush Maheshwari\\
  NetID: pm489\\
  Collaborators: None
}

\bigskip

\textbf{(2)}
\emph{(10 points)}
Solve Exercise 4.16 in Chapter 4 of the Kleinberg-Tardos textbook.

\emph{\textbf{Hint:}} The simplest proof of correctness that we know
of uses an exchange argument.

\emph{\textbf{Karma exercise:}} Show how to implement your algorithm in time $O(n \log n)$.


\section{Solution}
\subsection{Problem Statement}
Given a $N$ transactions where each transaction is represented by $t_i\pm e_i$ and a list of $N$ events represented by $x_i$, we have to match every event to a distinct transaction such that $|t_i - x_i| < e_i$. If this matching is possible output Yes, else output No.
\subsection{Algorithm}

The algorithm starts by first maintaining an count array A[] initialized to 0. At any point in the algorithm a[i] is equal to one if that event has been matched to an transaction and a[i] is equal to zero if that event hasn't been matched to any transaction. We maintain the transactions in a list where each list element points to another list of candidate events which could match with this transaction. Then go through the list of events sorted by time one by one, and for each transaction we check if it can be matched with this event. This can be done by checking $|t_i - x_i| < e_i$. If this is true then we add that event into the list of candidate events for this transaction. Then we sort the transaction list based on the number of events in its candidate list. Then we go through the sorted transaction list. For each transaction we choose an event $e_i$ which hasn't been picked yet. This check can be done by whether the value of a[i] is zero or not. Pick the first such element in the list and mark a[i] equal to 1. If no such element can be picked which is possible if the list if empty or all the elements in the list have been already been matched, output No. If we find a matching event for all the transactions in the sorted transaction list then output Yes. 

\begin{algorithm}[H]
  \caption{}
  \begin{algorithmic}[1]
	\Procedure{event\_matcher}{$T\lbrack \rbrack, E\lbrack\rbrack , X\lbrack\rbrack$}
		\Let{$A\lbrack\rbrack$}{${0}$} \Comment{ Count array for events}
		\State$i \gets 0$
		\State$ result \gets Yes$ \Comment{ Default value of result is True}
		\While{$i < N $}\Comment{Initialize list to contain candidate events for each transaction  $S$}
			\State$L[i]\gets (0, \lbrack\rbrack) $\Comment{Each element is the size and an empty list}
		\EndWhile
		\State $x \lbrack\rbrack \gets sorted(x\lbrack\rbrack)$ \Comment{Sort x[]}
		\ForAll{$x_i \in X\lbrack\rbrack$}
			\ForAll{$t_j ,  e_j \in T\lbrack\rbrack , E\lbrack\rbrack$}
				\If {$|t_j - x_j| < e_i$}
					\State Append i to $L[j][1]$ \Comment{ list[i][1] represents the event list of list[i]}
					\State $L[i][0] \gets L[i][0] + 1$ \Comment {Increase the size of the candidate event list}
				\EndIf
			\EndFor
		\EndFor
		\State $L[] \gets sorted(L[])$ \Comment{Sort l[i] based on size of list contained in list[i][0]}
		\ForAll{$l_i \in  L\lbrack\rbrack$}
			\State $found \gets false$
			\ForAll{$e_j \in l[i][1]$}
				\If {$ a[j] == 0$}
					\State found = true
					\State $a[j] = 1$
					\State break
				\EndIf
			\EndFor
			\If {$found == false$} \Comment{This transaction had no matching event}
				\State $result \gets No$
				\State break
			\EndIf
		\EndFor
		\State \Return{$result$}
	\EndProcedure
  \end{algorithmic}
\end{algorithm}

\subsection{Proof of Correctness}

\begin{lemma} If there exists no valid match solution to the problem then the method EVENT\_MATCHER would return No as output.
\begin{proof} There are two case when there would be no valid match to the problem
\begin{enumerate}
\item If there is no event which lies within a transaction - In this case the algorithm will return false since the check in line $21$ of the algorithm would fail for that particular transaction and return No
\item If \textbf{for any set $S$} of transaction the total number of distinct events that lie within the transactions in $S$ is less that the size of $S$ - Let S be any such set of size $N$. Assume the total number of distinct transactions that lie within $S$ is $X < N$ and let they be $e_1, e_2 ... e_x$. Since this case is different from the case above, this means that no transaction has empty event list, which means that the transactions have overlapping events. Every time we execute the for loop in line 14, we find a matching event to that transaction and mark it 1 in the count array a. Since $X<N$, this means that there must be some transaction j which would not be able to find a matching transaction. And hence the algorithm will return No on such a transaction.

Combining both case 1 and 2, we prove that if there is no valid match to the problem then the algorithm would return No
\end{enumerate}
\end{proof}
\end{lemma}

\begin{lemma}
If there is some transaction which contains a list of candidate events, then the algorithm matches the transaction with an event which has the least time stamp among the unmatched events in the candidate list for that transaction
\begin{proof}
First we observe that we sort the event list in step 7 of the algorithm. Hence when the events are matched with the transaction in steps $8-12$, they are added in the transaction list in a sorted order based on time stamp. Now when the for loop is running in lines $16-20$ to find an event to match with a transaction , it would pick the an unmatched event with the least time stamp since that list is sorted. By unmatched event, we mean any event which hasn't been matched to any transaction yet.
\end{proof}
\end{lemma}

\begin{lemma}
If a solution exists where each transaction is matched with an unmatched event in its list with the least time stamp, our algorithm will find it. 
\begin{proof}
This follows from lemma 2.
\end{proof}
\end{lemma}

\begin{lemma} If there exists a valid match solution to the problem then the method EVENT\_MATCHER would return Yes as output.
\begin{proof} 
We can divide the solution set into two cases and prove this lemma separately for both
\begin{enumerate}
\item The candidate event list for each transaction contains only one event. In this case there is only one output and our algorithm would find it and output Yes.

\item In this case there is some transaction which has more than one event in its candidate list. \\Let the solution where each transaction is matched with an unmatched event in its list with the least time stamp be $S_0$. 
We will prove that if there is a solution S which is different from the solution output by our algorithm $S_0$, we can transform it into $S_0$ ( which is our greedy solution). So we will try to show that any solution can be transformed into $S_0$ and we know from lemma 3 that if a solution $S_0$ exists our algorithm will find it. Hence we can complete the proof.
\\Also when we call a solution $S_a$ closer to $S_0$ than $S_b$ we mean that $S_a$ contains more transactions which are matched to their first(in timestamp order) unmatched event than $S_b$ does.\\

From lemma 2, we know that our algorithm matches each transaction to an unmatched event in its list which has the least time stamp. Now consider a solution $S$ in which an transaction $t_i$ has been matched to an event $e_b > e_a$ but $e_a$ was in its candidate list at that point and was unmatched. This means that there exists some solution $S'$ where $t_i$ had been matched to $e_a$ otherwise $e_a$ is not a valid unmatched candidate. Let in the solution $S$, $e_a$ is matched to some $t_j$. \\

Now there can two cases. In the first case $t_j $ can span $e_b$. Then we can simply swap the events of $t_i$ and $t_j$ and we get a solution which is strictly closer to $S_0$.\\
In the second case if $t_j$ does not span $e_b$, this means that $ t_j + e_j < e_b$.  Assume $t_j$ is matched to some $e_m$ in $S'$ such that $e_m < e_b$ since $ t_j + e_j < e_b$. Let this $e_m$ is matched to some $t_k$ in $S$. Now either $t_k + e_k < e_b$ or there exists some $e_n$ to which it was matched in $S'$ such that $e_n < e_b$. We can keep on using the same argument and since we have a solution with finite number of transactions, ultimately we will get a $t_g$ which spans across $e_b$ that is $t_g + e_g < e_b$. Now apply to following transformation to $S$ - match $t_j$ to $e_m$, $t_k$ to $e_n$ ... $t_g$ to $e_b$ and then we can finally match $t_i$ to $e_a$. We have basically found a sort of a circular matchings and just shifted the matched pairs such that they still have valid matches. We will still have a valid solution $S''$ which is closer to $S_0$ than $S$, since we have matched $t_i$ to an unmatched candidate event with the least time stamp. Now we can repeatedly use the same transformation, until the solution $S$ is completely transformed into $S_0$. And hence we have proved that if there exists a valid solution $S$ different from $S_0$, there will exists a valid solution $S_0$ which can be found by applying the above transformation.
\end{enumerate}

Combining lemma 3 and the argument above, we have proved that if there exists a valid solution, then out algorithm will find it. 

\subsection{Complexity Analysis}
\begin{enumerate}
\item Steps 5-6 take $O(n)$
\item Step 7 takes $O(nlogn)$
\item Step 8-12 takes $O(n*n)$
\item Step 13 takes $O(nlogn)$
\item Step 14-23 takes $O(n*n)$
\end{enumerate}

Total time is $O(n*n)$


\end{proof}
\end{lemma}
\end{document}
