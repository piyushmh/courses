\documentclass[11pt]{article}
%\usepackage{fullpage}
\usepackage{epic}
\usepackage{eepic}
\usepackage{paralist}
\usepackage{graphicx}
\usepackage{tikz}
\usepackage{xcolor,colortbl}

\usepackage{fullpage}
\usepackage{amsmath,amsthm,amssymb}
\usepackage{algorithmicx, algorithm}
\usepackage[noend]{algpseudocode}

\newcommand*\Let[2]{\State #1 $\gets$ #2}
\newtheorem{theorem}{Theorem}
\newtheorem{lemma}[theorem]{Lemma}
\newtheorem{proposition}[theorem]{Proposition}
\newtheorem{corollary}[theorem]{Corollary}

\newenvironment{definition}[1][Definition]{\begin{trivlist}
\item[\hskip \labelsep {\bfseries #1}]}{\end{trivlist}}
\newenvironment{example}[1][Example]{\begin{trivlist}
\item[\hskip \labelsep {\bfseries #1}]}{\end{trivlist}}
\newenvironment{remark}[1][Remark]{\begin{trivlist}
\item[\hskip \labelsep {\bfseries #1}]}{\end{trivlist}}

\newcommand*\Th[1]{{#1}^{\textrm{th}}}

%%%%%%%%%%%%%%%%%%%%%%%%%%%%%%%%%%%%%%%%%%%%%%%%%%%%%%%%%%%%%%%%
% This is FULLPAGE.STY by H.Partl, Version 2 as of 15 Dec 1988.
% Document Style Option to fill the paper just like Plain TeX.

\typeout{Style Option FULLPAGE Version 2 as of 15 Dec 1988}

\topmargin 0pt
\advance \topmargin by -\headheight
\advance \topmargin by -\headsep

\textheight 8.9in

\oddsidemargin 0pt
\evensidemargin \oddsidemargin
\marginparwidth 0.5in

\textwidth 6.5in
%%%%%%%%%%%%%%%%%%%%%%%%%%%%%%%%%%%%%%%%%%%%%%%%%%%%%%%%%%%%%%%%

\pagestyle{empty}
\setlength{\oddsidemargin}{0in}
\setlength{\topmargin}{-0.8in}
\setlength{\textwidth}{6.8in}
\setlength{\textheight}{9.5in}

\setcounter{secnumdepth}{0}

\setlength{\parindent}{0in}
\addtolength{\parskip}{0.2cm}
\setlength{\fboxrule}{.5mm}\setlength{\fboxsep}{1.2mm}
\newlength{\boxlength}\setlength{\boxlength}{\textwidth}
\addtolength{\boxlength}{-4mm}

\newcommand{\algobox}[2]{
  \begin{center}
    \framebox{\parbox{\boxlength}{
        \textbf{Introduction to Algorithms} \hfill \textbf{#1}\\
        \textbf{CS 4820, Spring 2014} \hfill \textbf{#2}}}
  \end{center}}

\newcommand{\algosolutionbox}[2]{
  \begin{center}
    \framebox{\parbox{\boxlength}{
        \textbf{CS 4820, Spring 2014} \hfill \textbf{#1}\\
        #2
      }}
  \end{center}}


\begin{document}

\algosolutionbox{Homework 6, Problem 3}{
  % TODO: fill in your own name, netID, and collaborators
  Name: Piyush Maheshwari\\
  NetID: pm489\\
  Collaborators: None
}

\bigskip

\textbf{(3)}
\emph{(5 points)}
{\em In this problem --- as in the lectures this semester,
but contrary to the textbook --- a flow network may
have edges entering the source and/or leaving the sink.}

A flow network is called {\em Eulerian} if the combined 
capacity of the incoming edges at any vertex is equal to 
the combined capacity of the outgoing edges at that vertex.
If $a$ and $b$ are any two distinct vertices of an Eulerian
flow network, prove that the maximum flow from $a$ to $b$ is
equal to the maximum flow from $b$ to $a$.

\bigskip
\subsection{Solution}
\begin{lemma}
In an Eulerian graph, for any cut $A,B$, we have C(A,B) = C(B,A) where C(A,B) is the capacity of all edges going from A to B.
\begin{proof}
Let $C(x,A)$ where $x$ is any vertex in the graph G and A is any subset of vertices in G denote the sum of capacities of all edges going from x to any vertex in $A$.

Let $C(A,x)$ be the sum of capacities of all edges going from any vertex in $A$ to vertex x.

Since we know that capacity going out from each node is equal to the capacity going in, we have $C(x,V) = C(V,x)$ where $V$ is the vertex set in G.

Now for any cut A,B we have $V=A+B$.

This means that 
C(x, A) + C(x,B) = C(A,x) + C(B,x) for any vertex x

Now let all the vertices in A be $a_1, a_2 ... a_{|A|}$. Then we have\\\\
$C(a_1, A) + C(a_1,B) = C(A,a_1) + C(B,a_1)$\\
$C(a_2, A) + C(a_2,B) = C(A,a_2) + C(B,a_2)$\\
..\\
..\\
$C(a_{|A|}, A) + C(a_{|A|},B) = C(A,a_{|A|}) + C(B,a_{A|})$\\\\
Adding all these equations

$C( (a_1 + a_2 .. + a_{|A|}), A) + C((a_1 + a_2 .. + a_{|A|}),B) = C(A,(a_1 + a_2 .. + a_{|A|})) + C(B,(a_1 + a_2 .. + a_{|A|}))$\\\\
Since $a_1 + a_2 .. + a_{|A|}$ = $A$ , we have \\\\
$C(A,A) + C(A,B) = C(B,A) + C(A,A)$\\\\
$C(A,B) = C(B,A)$
\end{proof}
\end{lemma}

\subsection{Proof}

Now let the maximum flow from a to b has a minimum cut $A_1,B_1$. From max flow min cut theorem we know that this is also the minimum cut among all cuts from a to b. Since $C(A_1,B_1)$ is also equal to $C(B_1, A_1)$ from lemma 1, this means that $C(B_1, A_1)$ is the minimum cut for all flows from b to a.

Now let the maximum flow from b to a has a minimum cut $B_2,A_2$. From max flow min cut theorem we know that this it min cut among all cuts from b to a. But we also know that $C(B_1, A_1)$ is the minimum cut from b to a. This means that \\\\
$C(B_2, A_2)$ = $C(B_1, A_1)$\\
$C(B_2, A_2)$ = $C(A_1, B_1)$\\\\
Also max flow from a to b = $C(A_1, B_1)$\\
max flow from b to a = $C(B_2, A_2)$\\\\
This means that max flow from a to b is equal to max flow from b to a.

\end{document}
