\documentclass[11pt]{article}
%\usepackage{fullpage}
\usepackage{epic}
\usepackage{eepic}
\usepackage{paralist}
\usepackage{graphicx}
\usepackage{tikz}
\usepackage{xcolor,colortbl}

\usepackage{fullpage}
\usepackage{amsmath,amsthm,amssymb}
\usepackage{algorithmicx, algorithm}
\usepackage[noend]{algpseudocode}

\newcommand*\Let[2]{\State #1 $\gets$ #2}
\newtheorem{theorem}{Theorem}
\newtheorem{lemma}[theorem]{Lemma}
\newtheorem{proposition}[theorem]{Proposition}
\newtheorem{corollary}[theorem]{Corollary}

\newenvironment{definition}[1][Definition]{\begin{trivlist}
\item[\hskip \labelsep {\bfseries #1}]}{\end{trivlist}}
\newenvironment{example}[1][Example]{\begin{trivlist}
\item[\hskip \labelsep {\bfseries #1}]}{\end{trivlist}}
\newenvironment{remark}[1][Remark]{\begin{trivlist}
\item[\hskip \labelsep {\bfseries #1}]}{\end{trivlist}}

%%%%%%%%%%%%%%%%%%%%%%%%%%%%%%%%%%%%%%%%%%%%%%%%%%%%%%%%%%%%%%%%
% This is FULLPAGE.STY by H.Partl, Version 2 as of 15 Dec 1988.
% Document Style Option to fill the paper just like Plain TeX.

\typeout{Style Option FULLPAGE Version 2 as of 15 Dec 1988}

\topmargin 0pt
\advance \topmargin by -\headheight
\advance \topmargin by -\headsep

\textheight 8.9in

\oddsidemargin 0pt
\evensidemargin \oddsidemargin
\marginparwidth 0.5in

\textwidth 6.5in
%%%%%%%%%%%%%%%%%%%%%%%%%%%%%%%%%%%%%%%%%%%%%%%%%%%%%%%%%%%%%%%%

\pagestyle{empty}
\setlength{\oddsidemargin}{0in}
\setlength{\topmargin}{-0.8in}
\setlength{\textwidth}{6.8in}
\setlength{\textheight}{9.5in}

\setcounter{secnumdepth}{0}

\setlength{\parindent}{0in}
\addtolength{\parskip}{0.2cm}
\setlength{\fboxrule}{.5mm}\setlength{\fboxsep}{1.2mm}
\newlength{\boxlength}\setlength{\boxlength}{\textwidth}
\addtolength{\boxlength}{-4mm}

\newcommand{\algobox}[2]{
  \begin{center}
    \framebox{\parbox{\boxlength}{
        \textbf{Introduction to Algorithms} \hfill \textbf{#1}\\
        \textbf{CS 4820, Spring 2014} \hfill \textbf{#2}}}
  \end{center}}

\newcommand{\algosolutionbox}[2]{
  \begin{center}
    \framebox{\parbox{\boxlength}{
        \textbf{CS 4820, Spring 2014} \hfill \textbf{#1}\\
        #2
      }}
  \end{center}}


\begin{document}

\algosolutionbox{Homework 9, Problem 2}{
  % TODO: fill in your own name, netID, and collaborators
  Name: Piyush Maheshwari\\
  NetID: pm489\\
  Collaborators: None
}

\medskip


\textbf{2.}
\emph{(10 points)}
Recall the halting problem:
An instance of the halting problem is a string of the form $x\# y$ and the goal is to decide if the Turing machine $M_x$ encoded by the string $x$ halts on input $y$.
(If $M_x$ halts on $y$, then $x\# y$ is a \textsc{yes} instance.
Otherwise, $x\# y$ is a \textsc{no} instance.)

Show that every Turing machine fails to solve the halting problem on an infinite number of instances.
(A Turing machine $M$ \emph{fails to solve} the halting problem for instance $x\#y$ if it loops on input $x\# y$ or if it produces the wrong answer, i.e., it rejects in case that $x\#y$ is a \textsc{yes} instance or it accepts in case that $x\# y$ is a \textsc{no} instance.)

\subsection{Solution}
\begin{proof}
We will sketch this proof by contradiction. Suppose that there exists some turing machine M which fails to solve the Halting Problem on a finite number of instances. Let the set of values on which M fails be $S$. Let the set $B$ = U - \{S\} which is basically the set of all problems not in B (U stands for universe of all problems)

Consider another TM $M'$ which basically checks if a given input is present in the $S$ or not. Using $M'$ and $M$ we can solve all problems not in S. Now consider the problems in S. We know that M fails on each of the problem in this set, but we don't know if M loops or gives an incorrect answer. But since S is a finite set, we know that there exists a mapping from each element s $\in$ S to \{\textsc{yes}, \textsc{no}\} where \textsc{yes} means that s is a \textsc{yes} instance of the Halting problem and \textsc{no} means that s is a \textsc{no} instance of the Halting problem. This means that if the cardinality of $S$ is $n$, there exists $2^n$ such mappings, each of which can be represented by a turing machine. We also know that since each problem either halts or not, one of these $2^n$ mappings (turing machines) would correctly classify each instance s $\in$ S. Let that turing machine be $M''$. We don't know how to find such a turing machine, but we definitely know that one such machine would exist since there are only a finite number of mappings.

Now build another turing machine H which given an input first checks if that input belongs to $S$ or not by running machine $M'$. Then if that input does not belong to $S$, it runs that input of machine $M$ to find the correct answer ( M will definitely halt). Otherwise if the input does not belong to the set $S$, run machine $M''$ and output the answer ($M''$ will definitely halt as well). \\\\
Now using machine H we have solved the Halting problem. But we know Halting problem is un-decidable (proved in class) and there does not exist a turing machine which can solve the halting problem for all input instances. This means that we have arrived at a contradiction and our initial assumption about the existence of $M$ was false. Hence every Turing Machine fails to solve the halting problem on infinite number of instances.
\end{proof}

\end{document}
