\documentclass[11pt]{article}
%\usepackage{fullpage}
\usepackage{epic}
\usepackage{eepic}
\usepackage{paralist}
\usepackage{graphicx}
\usepackage{tikz}
\usepackage{xcolor,colortbl}

\usepackage{fullpage}
\usepackage{amsmath,amsthm,amssymb}
\usepackage{algorithmicx, algorithm}
\usepackage[noend]{algpseudocode}

\newcommand*\Let[2]{\State #1 $\gets$ #2}
\newtheorem{theorem}{Theorem}
\newtheorem{lemma}[theorem]{Lemma}
\newtheorem{proposition}[theorem]{Proposition}
\newtheorem{corollary}[theorem]{Corollary}

\newenvironment{definition}[1][Definition]{\begin{trivlist}
\item[\hskip \labelsep {\bfseries #1}]}{\end{trivlist}}
\newenvironment{example}[1][Example]{\begin{trivlist}
\item[\hskip \labelsep {\bfseries #1}]}{\end{trivlist}}
\newenvironment{remark}[1][Remark]{\begin{trivlist}
\item[\hskip \labelsep {\bfseries #1}]}{\end{trivlist}}

%%%%%%%%%%%%%%%%%%%%%%%%%%%%%%%%%%%%%%%%%%%%%%%%%%%%%%%%%%%%%%%%
% This is FULLPAGE.STY by H.Partl, Version 2 as of 15 Dec 1988.
% Document Style Option to fill the paper just like Plain TeX.

\typeout{Style Option FULLPAGE Version 2 as of 15 Dec 1988}

\topmargin 0pt
\advance \topmargin by -\headheight
\advance \topmargin by -\headsep

\textheight 8.9in

\oddsidemargin 0pt
\evensidemargin \oddsidemargin
\marginparwidth 0.5in

\textwidth 6.5in
%%%%%%%%%%%%%%%%%%%%%%%%%%%%%%%%%%%%%%%%%%%%%%%%%%%%%%%%%%%%%%%%

\pagestyle{empty}
\setlength{\oddsidemargin}{0in}
\setlength{\topmargin}{-0.8in}
\setlength{\textwidth}{6.8in}
\setlength{\textheight}{9.5in}

\setcounter{secnumdepth}{0}

\setlength{\parindent}{0in}
\addtolength{\parskip}{0.2cm}
\setlength{\fboxrule}{.5mm}\setlength{\fboxsep}{1.2mm}
\newlength{\boxlength}\setlength{\boxlength}{\textwidth}
\addtolength{\boxlength}{-4mm}

\newcommand{\algobox}[2]{
  \begin{center}
    \framebox{\parbox{\boxlength}{
        \textbf{Introduction to Algorithms} \hfill \textbf{#1}\\
        \textbf{CS 4820, Spring 2014} \hfill \textbf{#2}}}
  \end{center}}

\newcommand{\algosolutionbox}[2]{
  \begin{center}
    \framebox{\parbox{\boxlength}{
        \textbf{CS 4820, Spring 2014} \hfill \textbf{#1}\\
        #2
      }}
  \end{center}}


\begin{document}

\algosolutionbox{Homework 9, Problem 1}{
  % TODO: fill in your own name, netID, and collaborators
  Name: Piyush Maheshwari\\
  NetID: pm489\\
  Collaborators: None
}

\medskip


\textbf{1.}
\emph{(10 points)}
The class \textbf{NP} is defined in terms of polynomial-time verifiers and polynomial-length certificates.
In this exercise, we will see that semidecidable problems can be characterized in terms of verifiers and certificates of unbounded length.
(A problem is called \emph{semidecidable} if the set of \textsc{yes} instance is recursively enumerable.)

Show that a problem $A$ is semidecidable if and only if there exists a Turing machine~$V$ that halts on every input and satisfies the following property:
a string $x$ is a \textsc{yes} instance of $A$ if and only there exists some string $y$ such that $V$ on input~$x\#y$ accepts.

\subsection{Solution}

Given a problem A which is semidecidable we know that there exists a turing machine $M$ which accepts x if x $\in$ A \\\\
Construct a turing machine $V$ which is basically the same machine $M$ which when given a input of the form x\#y simulates x on M for y steps. If M does not halt within y steps, V just halts and rejects. It accepts if M accepts x within y steps. If the input does not contain a finite y, then the $V$ halts.

We will show below that $V$ halts on every input and satisfies the property mentioned in the question.

\subsubsection{Proof of Correctness}
\begin{proof}
First we have to prove that the machine V that we have constructed halts on every input. We know that the machine V runs for maximum y steps. If M accepts before that, V also accepts. Otherwise V halts and rejects. Also if the value of y is not a finite valid value, then $V$ simply halts. Hence V halts on any input.
\end{proof}
\begin{proof}
Now we prove that given a semi decidable problem A, there exists such a turing machine $V$ which satisfies the property that it accepts on input x\#y if x is an \textsc{yes} instance of $A$. \\We know the machine M described above exists since A is semi decidable (by definition of semi decidable problem). Since V is a modified version of M, it also exists. Also we know that a \textsc{yes} instance of x is accepted by the machine M in a finite number of steps. Now if we run V on inputs x\#y by varying y from 1,2,..$\infty$, we know that we will find a finite value of y on which V accepts and halts. This value would be the number of steps that M takes to accept the \textsc{yes} instance x. Hence if $A$ is semi decidable then turning machine $V$ exists which halts on every input and satisfies the given property.\\\\
Now we prove the reverse direction of statement - if such a machine $V$ exists which follows this property, then A is semi decidable. Now we know that if $V$ accepts a input x\#y, this means that the string x is an \textsc{YES} instance of $A$ (using the iff condition of this property). We also know that $V$ halted within y number of steps. This means that if there is a \textsc{yes} instance of the problem $A$, we can construct a TM $M$ (which is basically V), which accepts x in a finite number of steps (which is y). This means that $A$ is semi decidable.

This completes the proof.
\end{proof}
\end{document}
